\documentclass{resume}

\usepackage[utf8]{inputenc}
\usepackage[english]{babel}
\usepackage{hyperref}
\hypersetup{
    colorlinks=true,
    linkcolor=blue,
    filecolor=magenta,      
    urlcolor=cyan,
}
\usepackage[left=0.75in,top=0.6in,right=0.75in,bottom=0.6in]{geometry} % Document margins
\newcommand{\tab}[1]{\hspace{.2667\textwidth}\rlap{#1}}
\newcommand{\itab}[1]{\hspace{0em}\rlap{#1}}
\name{Zihuang Guo} % Name
\address{33 W Ontario St, Chicago, IL 60654} % Address
\address{(+1) 2175502096 \mid guo605150172@gmail.com \mid linkedin.com/in/zihuang$-$guo \mid github.com/zgjohnny} % Phone number and email

\begin{document}

%----------------------------------------------------------------------------------------
%	EDUCATION SECTION
%----------------------------------------------------------------------------------------
\begin{rSection}{Education}

\\{\bf University of Illinois at Urbana-Champaign} \hfill {\em Aug 2016 $-$ Dec 2019} 
\\ Bachelor of Arts in Linguistic\hfill { GPA: 3.32 / 4.0 } \\
Minor in Computer Science
\end{rSection}

%--------------------------------------------------------------------------------
%  PROJECT SECTION
%-----------------------------------------------------------------------------------------------
\begin{rSection}{Projects}

\\{\bf Grade Disparity Between Sections at UIUC} \\
Worked with a group of students and a professor to create a data visualization for grade disparity between sections of every course at University of Illinois at Urbana-Champaign. Primarily worked on creating tool tips with metadata about each course. You can visit the project at \url{http://waf.cs.illinois.edu/discovery/grade_disparity_between_sections_at_uiuc/} for more details.

\end{rSection}

%----------------------------------------------------------------------------------------
%	WORK EXPERIENCE SECTION
%----------------------------------------------------------------------------------------
\begin{rSection}{Work Experience}

\begin{rSubsection}{Koya DocuTracker LLC, Chicago, IL} {\textit{May 2018 $-$ Present}}
{\normalfont{Software Engineer Intern}}

    \item Communicate with clients directly and make bug fixes, feature updates, performance increases and such to the application accordingly
    \item Lead and contribute to adding new features, debugging, testing automation, improving performance, and refactoring on the front-end
    \item Create and modified data structures on the back-end including database and NuGet packages, managed endpoints
    \item Create migrations and scripts necessary for parts of the application
    \item Design parts of user authorization and data authentication on the back-end
    \item Make hotfixes and releases for the company

\end{rSubsection}

\end{rSection}

%----------------------------------------------------------------------------------------
% SKILL SECTION
%----------------------------------------------------------------------------------------
\begin{rSection}{Skills}

\begin{itemize}

    \item   \textbf{Mandarin(Native)} and \textbf{English(Native or bilingual proficiency)}
    \item	Advanced in \textbf{JavaScript, TypeScript, C\#, Python, HTML, CSS, SCSS, C, C\boldsymbol{++}, Java}, proficient in \textbf{SQL, Verilog}, familiar with \textbf{Swift, MATLAB, R}
    \item	Advanced in \textbf{Node JS} and other Front-end technologies such as \textbf{Aurelia Framework}, \textbf{React}
    \item   Advanced in version Control (\textbf{Git} and \textbf{SVN})
    \item	Proficient in \textbf{Entity Framework, ASP.NET, Microsoft SQL Server}, farmiliar with \textbf{Microsoft Azure}
    \item	Proficient in automated testing applications such as \textbf{Katalon Studio}	
    \item   Proficient in data Analysis and Data Visualization (\textbf{pandas, scipy, d3})
    \item	Proficient in some AI Algorithms (\textbf{Searches, Bayes, TD-Learning, Reinforcement Learning, SARSA, Neural Network})
    \item	Proficient in Natural Language Processing (\textbf{scikit-learn, scipy, NLTK, Weka})
    \item	Familiar with system programming (\textbf{Multithreading, Synchronization, Inter-process Communication, Signaling, Networking})

\end{itemize}

\end{rSection}

\end{document}
